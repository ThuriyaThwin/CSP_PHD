\documentclass[]{article}
\usepackage{setspace,geometry}

\geometry{bindingoffset=0.4in,margin=1.2in}

\title{Progress Report Towards PhD Degree, Year Three\\
  Research Topic: Rational Metareasoning in Problem Solving Search}
\author{David Tolpin\\
        Department of Computer Science\\
        Ben-Gurion University of the Negev\\
        \\
        Under the supervision of Professor Solomon Eyal Shimony\\
        \\
      \\}

\doublespacing

\begin{document}

\maketitle

\section{Research Progress}

During the third year of my Ph.D. studies I worked on Monte Carlo Tree
Search, in particular on VOI-aware MCTS for simple regret.

MCTS, and especially UCT \cite{Kocsis.uct} appears in numerous search
applications, such as \cite{Eyerich.ctp}. Although these methods are
shown to be successful empirically, most authors appear to be using
UCT ``because it has been shown to be successful in the past'', and
``because it does a good job of trading off exploration and
exploitation''. While the latter statement may be correct for the
Multi-armed Bandit problem and for the UCB1 algorithm \cite{Auer.ucb},
we argue that a simple reconsideration from basic principles can
result in schemes that outperform UCT.

The core issue is that in MCTS for adversarial search and search in
``games against nature'' the goal is typically to find the best
first action of a good (or even optimal) policy, which is closer to
minimizing the simple regret, rather than the cumulative regret
minimized by UCB1.  However, the simple and the cumulative regret
cannot be minimized simultaneously; moreover, \cite{Bubeck.pure} shows
that in many cases the smaller the cumulative regret, the greater the
simple regret.

In a series of
papers~\cite{TolpinShimony.mcts,HayRussellTolpinShimony.selecting,TolpinShimony.voiaware},
we proposed a new MCTS sampling scheme, SR+CR, which is based on
optimizing the simple regret of sampling, and outperforms UCT
empirically. Further on, we derived theoretical
distribution-independent bounds for the blinkered VOI estimate, and
improved the sampling scheme through the use of the bound-based VOI
estimate. The new sampling scheme was emprically evaluated on a number
of domains, including POMDP and Computer Go, and outperformed other
known sampling schemes.

The research was presented at the leading conferences in the field
of Artificial Intelligence: AAAI-2012, UAI-2012, ECAI-2012.

\section{Publications}

\bibliographystyle{plain}
\bibliography{refs}

\end{document}
