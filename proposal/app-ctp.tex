In the {\em Canadian Traveler Problem} (CTP) \cite{Barnoy.ctp} a
traveling agent is given an undirected connected weighted graph
$G=(V,E)$ as input together with its initial source vertex ($s \in
V$), and a target vertex ($t \in V$).  The input graph $G$ may undergo
changes, that are not known to the agent, {\em before} the agent
begins to act, but remains fixed subsequently.  In particular, some of
the edges in $E$ may become {\em blocked} and thus untraversable. Each
edge $e$ in $G$ has a weight, or cost, $w(e)$, and is blocked with a
probability $p(e)$, where $p(e)$ is known to the agent. The agent can
perform {\em move} actions along an unblocked edge which incur a {\em
  travel cost} $w(e)$.  Traditionally, the CTP was defined such that
the status of an edge can only be revealed upon arriving at a node
incident to that edge, i.e., only {\em local sensing} is allowed.

A somewhat more general version of CTP is {\em CTP with sensing}. In
this variant, in addition to move actions (and local sensing), an
agent situated at a vertex $v$ can also perform {\em sense} actions
and query the status of any edge $e \in G$. The sensing action incurs a
{\em sensing cost} $sc(v,e)$. The task of the agent is to travel to
the goal while minimizing a total cost
$C_{total}=C_{travel}+C_{sensing}$.

In a general case, value of information of a sensing action can be
efficiently estimated only under rather strong simplifying
assumptions, such as the free-space assumption
\cite{Bnaya.sensing}. Special cases in which a better VOI estimate can
be computed efficiently are of interest.

In one such setting, the sensing costs are much smaller than the
travel costs, and it makes sense to ensure that the path to be
traversed is open before beginning traversal.  In this variant, called
{\em CTP-Sensing-First} (CTP-SF), the problem is to minimize expected
sensing costs in order to detect the shortest open path. The
base-level action uses a polynomial-time algorithm (such as the
Dijkstra algorithm for the single-source shortest path problem
\cite{Kleinberg.algorithms}) to find the shortest path between the
source and the target nodes.

\subsection{Proposed research}
\label{sec:app-ctp-research}

An algorithm based on rational
metareasoning can be used to solve the CTP-SF problem, which can be
formulated as follows:
\begin{itemize}
\item The application of a polynomial-time CTP algorithm to the CTP
  graph updated according to the remote sensing outcomes is the only
  base-level action.
\item The travel cost of the algorithm, taken with the negative sign,
  corresponds to the utility of the base-level action.
\item The remote sensing actions, performed before the traditional CTP
  algorithm is applied, correspond to the computational actions.
\end{itemize}
This problem can be solved approximately using weaker simplifying
assumptions than for the general case of CTP with sensing, since the
sensing and the traversal are not interleaved, and the net utility of
the base-level action is easily computable. Optimal solutions for
restricted cases of the problem with particular graph topology may be
discovered. Finally, the algorithm for solving the CTP-SF problem
can be used to design an heuristic for solving the general CTP problem
with sensing.