
\documentstyle[10pt]{letter}
\textheight 10in
\textwidth 6.8 truein
\voffset -2.0cm
\hoffset -2.0cm

\begin{document}

%\vspace*{-0.5in}

\name{Solomon Eyal Shimony}

\signature{Prof. Solomon Eyal  Shimony\\
Vice-Chair\\
Department of Computer Science\\
Ben-Gurion University}



\begin{letter}{  Reference letter for Mr. David Tolpin}

\opening{Dear  Award Commitee Member, }

I am very happy to write this recommendation letter
on behalf of David Tolpin, whom
I have known for about 4 years, first as an MSc supervisor
and now as his PhD supervisor.

Immediately upon his arrival we were impressed with David's
deep understanding of new material, and elected to employ him
in the IMG4 consortium. His role was
to research and develop algorithms and software for optimizing sensing plans
for optimization under uncertainty, resulting from such tasks
occuring in processes used by the industrial partners at IMG4.
He has defined his own niche in the optimization problem and
contributed truly novel ideas of both theoretical and practical scientific
significance. 

This early work included the definition and 
heuristic solution of the selection problem (which also has applications
in meta-reasoning), on which
he has published a joint landmark paper on the selection problem
in the highly ranked IEEE-SMC-B as well as another accepted 
journal paper on efficient 
estimation of value-of-information for optimization problems.
Meta-reasoning for search, on which he focuses,
is a field which has re-emerged most strongly due to the
recent huge success of Monte-Carlo tree search (MCTS) and the role
meta-reasoning takes in these methods. 
His more recent work has been published or accepted for publication
at venues of the highest order: IJCAI 2011, and a paper
accepted to AAAI 2012, both the top conferences
in AI. Additional publications
were submitted to other high-quality conferences (UAI-2012, ECAI-2012).
In all these publications (joint with me) David did all empirical
work, most of the theroetical work, and usually more
than half the actual writing. Some of the latest submitted joint
pending papers are practically solo work, involving only final polishing
touches by the advisor.

In addition to doing exceptional research, David
has taken on a leadership role. He has in the past defined the 
engineering parameters of a demo system including
both his algorithms and those developed by other team members, and has
incorporated them into a superb
complete system that has been very appreciated by the
industrial partners and Ministry of Trade and Industry evaluators.
He has also shown capability for multi-disciplinary work, as is also evident
from published work and patents that he has done before arriving at BGU.
His research to-date at BGU  is already more than sufficient
as a PhD dissertation in computer science, even though he is 
still in the middle of his third year as
a PhD student.

Achieving all of the above, together with TA work in which he was 
extremely methodical
and an indispensible contribution to the department (especially the 
SP lab course for Bioinformtics, which he does solo), is nothing
short of amazing. He is the
best PhD student that I have ever had as a solo advisor,
roughly on par w.r.t. research quality
with Carmel Domshlak, my top past (joint) PhD student
(who is now a very successful senior faculty member at the Technion).

The sum total of the above make him a top PhD student.
I therefore most strongly and 
unequivocally recommend David Topin for the
Friedman Award.


\closing{Yours Sincerely,}
\end{letter}


\end{document}


