\documentclass{article}

\usepackage{enumerate}
\usepackage{url}
\usepackage{amsfonts,amsmath,amsthm,amssymb}
\usepackage{color}
\usepackage{geometry}

\geometry{margin=1.2in}

\title{Research Summary}
\author {David Tolpin,
Department of Computer Science, \\
Ben-Gurion University of the Negev, Beer Sheva, Israel \\
tolpin@cs.bgu.ac.il}

\begin{document}

\maketitle

An important field of Artificial Intelligence is
problem-solving search. In problem-solving search, a single agent acts
in a neutral environment to reach the goal. Many problems, such as
routing and path-finding problems, finite-domain constraint
satisfaction, function optimization, fit within the problem-solving
search abstraction. General search algorithms capable of solving large
sets of search problems are well known. Algorithm performance can be
significantly improved by tuning the algorithm to a particular problem
domain; however, such fine-tuned algorithms exhibit good performance
only on small sets of search problems, and the effort invested in the
algorithm design cannot be reused in other problem domains.

Specialized versions of general search algorithms are often created by
combination and selective application of search  heuristics.
A human expert decides which heuristics to use with the
problem domain, and specifies how the search algorithm should apply
the heuristics to solve a particular problem instance. A search
algorithm that rationally selects and applies heuristics would decrease
the need for the costly human expertise. Principles of rational
metareasoning can be used to design rational search agents.

Some rational search algorithms were designed and shown to compare to
or even outperform manually tuned algorithms. However, wide adoption of
rational metareasoning algorithms for problem solving search is hindered
both by theoretical difficulties and by lack of problem domain specific
case studies. This research aimed at lifting some of the theoretical
difficulties in application of the rational methodology, as well as
at devising rational metareasoning versions of
algorithms for applications of problem-solving search.
The research topics were presented in a number of publications in 
refereed journals and proceedings of leading conferences on Artificial Intelligence:
\begin {itemize}
\item Semimyopic measurement selection for optimization under
  uncertainty~\cite{TolpinShimony.blinkered};
\item Rational value of information estimation for measurement
		  selection~\cite{TolpinShimony.raticomp};
\item Adaptive deployment of value-ordering heuristics in constraint
  satisfaction problems~\cite{TolpinShimony.csp};
\item Monte Carlo tree search based on simple regret~\cite{TolpinShimony.mcts,HayRussellTolpinShimony.selecting};
\item Decreasing heuristic evaluation time in a variant of A*~\cite{TolpinEtAl.rla}. 
\end {itemize}
In addition to the algorithm improvements, the studies demonstrated a
number of common rational metareasoning techniques which can be 
extended to other problem types. In particular,
\begin{itemize}
\item \cite{TolpinShimony.blinkered}, a journal paper published inthe
  highly ranked IEEE-SMC-B, includes definition and heuristic 
  solution of the selection problem, as well as introduces a less
  restricting and more powerful value of information estime --- the
  blinkered estimate --- used in later publications as well as employed by other
  researchers.
\item \cite{TolpinShimony.raticomp}, another journal paper, describes a novel and
  efficient value of information estimation scheme for optimization problems. 
\item \cite{HayRussellTolpinShimony.selecting} provides
  distribution-independent upper bounds for semi-myopic VOI estimates in Monte-Carlo sampling.
\item \cite{TolpinEtAl.rla} introduces a novel area of application of rational
metareasoning---optimal search in optimization problems.
\end{itemize}

Aspects of methodology of rational metareasoning initially
presented in the conference
publications~\cite{TolpinShimony.mcts,HayRussellTolpinShimony.selecting}
on VOI-aware Monte Carlo Tree Search and~\cite{TolpinEtAl.rla}
on rational deployment of heuristics in A* are, despite
significant theoretical and empirical results, just the first
steps in the exploration of the corresponding research fields.
Some important results had to be omitted due to inevitable space
limitations of conference proceedings.  Additional work which
has been and being done in the above-mentioned research directions
justifies publication of extended journal versions of the papers
with new theoretical results and expanded empirical evaluations.  

In particular, the advances in Monte Carlo Tree Search are based on
distribution-independent VOI bounds. In the publication~\cite{HayRussellTolpinShimony.selecting}, the improved
sampling scheme had been empirically evaluated on the domain of
Computer Go.  Another domain where MCTS recently proved to be
efficient is online planning in Markov Decision Processes. Some
results for MCTS in MDP were obtained, but not yet published.
The results address both a different family of distributions
appearing in the planning problems, and empirical evaluation
of the scheme on additional domains. Recent work of other authors
on sampling-based planning~\cite{FeldmanDomshlak.onlinemdp,DomshlakFeldman.uctornot}
provides additional insights and should be referenced in the
analysis. 

The published work on A*~\cite{TolpinEtAl.rla} reports
results on rational lazy deployment
of two heuristics in Lazy A*, a variant of A*. Only some results
of the theoretical analysis were published. In addition, the same
novel approach can be applied to A* when only a single heuristic
available, and to another variant of A*, IDA*, in the presence of
multiple heuristics. Work on new theoretical results, analyzing 
rational selection of the order of heuristic, and of the most
appropriate deployment scheme, is underway. Empirical evaluation
of the heuristic deployment scheme on additional domain families
and combination of heuristic was done and can be further usefully
extended, but has not yet been published.

\maketitle
\bibliographystyle{plain}
\bibliography{refs}

\end{document}
