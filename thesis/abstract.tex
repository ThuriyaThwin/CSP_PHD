An important class of Artificial Intelligence problems is
problem-solving search. In problem-solving search, a single agent acts
in a neutral environment to reach the goal. Many problems, such as
routing and path-finding problems, finite-domain constraint
satisfaction, function optimization, fit within the problem-solving
search abstraction. General search algorithms capable of solving large
sets of search problems are well known. Algorithm performance can be
significantly improved by tuning the algorithm to a particular problem
domain; however, such fine-tuned algorithms exhibit good performance
only on small sets of search problems, and the effort invested in the
algorithm design cannot be reused in other problem domains.

Specialized versions of general search algorithms are often created by
combination and selective application of search  heuristics.
A human expert decides which heuristics to use with the
problem domain, and specifies how the search algorithm should apply
the heuristics to solve a particular problem instance. A search
algorithm that rationally selects and applies heuristics would decrease
the need for the costly human expertise. Principles of rational
metareasoning can be used to design rational
search agents.

Some rational search algorithms were designed and shown to compare to
or even outperform manually tuned algorithms. However, wide adoption
of rational metareasoning algorithms for problem solving search is
hindered both by theoretical difficulties and by lack of problem
domain specific case studies. This research aims at lifting some of
the theoretical difficulties in application of the rational
methodology. In particular, the problem of efficiency estimating
the value of information of computational actions is considered. In
addition, this research proposes rational metareasoning versions of
algorithms for applications of problem-solving search in the areas of
parameter tuning, constraint satisfaction, canadian traveller problem,
and Bayesian network structure learning.
