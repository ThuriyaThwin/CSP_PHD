In this chapter three case studies of applications of rational
metareasoning are presented. The studies explore rational
metareasoning in the context of satisfaction problems
(Section~\ref{sec:cs-csp}), as well as of approximating
(Section~\ref{sec:cs-mcts}) and optimal (Section~\ref{sec:cs-mcts})
search in optimization problems. In addition to providing a
metareasoning-based solution to a particular problem, the studies
outline both techniques which are common to different problem types
(identifying base-level and computational actions, 
defining the utility and the value of information, etc.)
and those which exploit features present only in some of the problems,
such as estimating the VOI based on distribution-independent bounds
(Section~\ref{sec:cs-mcts}). 

A straightforward application of rational metareasoning often faces
difficulties because some of the assumptions of the approach do not
hold. For example, estimating the value of information depends on
the belief distribution of action outcomes; however, a reasonable choice
of the distribution is not always possible. Section~\ref{sec:cs-csp}
derives the distribution model from the properties of the search
algorithm rather than from the set of problem instances. Another
example is the myopic assumption \cite{Russell.right} which suggests
that an action can be chosen based on its anticipated immediate
effect. In some search algorithms, the information is obtained through
\emph{sampling}---recurring actions with probabilistic outcomes---and
the myopic VOI of a single sample is often
zero. Section~\ref{sec:cs-mcts} addresses such a case in the context
of Monte Carlo Tree Search and suggests a semi-myopic VOI estimate.

Advanced search algorithms are often parameterized and require tuning
to achieve the best performance. Building an algorithm upon the
principles of rational metareasoning allows to keep the number of
tunable parameters to a minimum---a single parameter needs to be tuned
in Sections~\ref{sec:cs-csp},~\ref{sec:cs-rla}. In addition, as
Section~\ref{sec:cs-csp} demonstrates, parameters
of a metareasoning-based algorithm are likely to reflect
implementation details of the algorithm and the heuristics rather
than features of a particular set of problem instances.

Finally, rational metareasoning is commonly applied to approximating
search where decision quality is traded for computation time. Optimal
search algorithms aim at finding the optimal solution in the shortest
possible time, and seemingly little can be achieved through
application of rational metareasoning.  Section~\ref{sec:cs-rla}
presents metareasoning-based improvements to a variant of
the~\astar~algorithm which result in finding the optimal solution in a
shorter time using the same heuristic functions; the employed
technique is general enough to extend to other optimal
search algorithms.


