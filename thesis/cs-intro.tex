In this chapter three case studies of applications of rational
metareasoning are presented. The studies explore rational
metareasoning in the context of satisfaction problems
(Section~\ref{sec:cs-csp}), as well as of approximating
(Section~\ref{sec:cs-mcts}) and optimal (Section~\ref{sec:cs-mcts})
search in optimization problems. In addition to providing a
metareasoning-based solution to a particular problem, the studies
outline both techniques which are common to different problem types
(identifying base-level and computational actions, 
defining the utility and the value of information, etc.)
and those which exploit features present only in some of the problems,
such as estimating the VOI based on distribution-independent bounds
(Section~\ref{sec:cs-mcts}). 

A straightforward application of rational metareasoning often faces
difficulties because some of the assumptions of the approach do not
hold. For example, estimating the value of information depends on
belief distribution of action outcomes; however, a reasonable choice
of the distribution is not always possible. Section~\ref{sec:cs-csp}
derives the distribution model from the properties of the search
algorithm rather than from the set of problem instances. Another
example is the myopic assumtion \cite{Russell.right} which suggests
that an action can be chosen based on its anticipated immediate
effect. In some search algorithms, the information is obtained through
\emph{sampling}---recurring actions with probabilistic outcomes---and
the myopic VOI of a single sample is often
zero. Section~\ref{sec:cs-mcts} addresses such a case in the context
of Monte Carlo Tree Search and suggests a semi-myopic VOI estimate.

[algorithm rather than problem set specific learning]

[VOI in optimal search]


