
%Since $t_2$ is often greater than $t_1$ additional search time may be saved if
%we can reduce the number of evaluations of $h_2$, even if this results in
%additional expansions and evaluations of $h_1$.
%We not present an algorithm which attempts to
%An algorithm attempting
%optimally manage such tradeoffs, which we call \textit{Rational Lazy A*}.
%is called \textit{Rational Lazy A*}.

Using the principles of rational meta-reasoning \cite{RussellWefald},
theoretically every computational operator (heuristic function evaluation, node
expansion, open list operation) should be treated as an action in a sequential
decision-making meta-level problem, and actions should be chosen so as to
achieve the minimal expected search time. However, the appropriate
general meta-reasoning problem is extremely hard to define precisely, and even harder
to solve optimally.

Therefore, we focus here on just one decision type, to be
made in the context of \lazyastar, when $n$ re-emerges from \OPEN~(Line 8).
We have two options: {\bf (1)}  Evaluate the second heuristic $h_2(n)$ and add
the node back to \OPEN~(Lines 8-10) like \lazyastar, or {\bf 2} bypass the
computation of $h_2(n)$ and expand $n$ right way (Lines 11 -13), thereby
saving timeby not computing $h_2$, at the risk of additional expansions and evaluations of $h_1$.
An method which attempts to
optimally manage this tradeoff, which we call \textit{Rational Lazy A*}, is presented next.
In order to choose rationally we define a criterion based on value of
information (VOI) of evaluating $h_2(n)$ in this context.

\subsection{VOI of evaluating $h_2$}

The only addition of RLA* to LA* is the option to bypass $h_2$ computations (Lines 8-10).
Suppose that we choose to compute $h_2$ --- this results in one of the
following outcomes:
\begin{enumerate}
\item $n$ is eventually expanded.
% EK: removed unnecessary detail
% : either immediately because $n$ still has the smallest $f$-cost,
% or it now has a higher $f$-cost, is re-inserted into \OPEN, but is still expanded later on.
\item $n$ is re-inserted into \OPEN, and the goal is found without ever expanding $n$.
\end{enumerate}

Observe that computing $h_2$ can be beneficial only in outcome 2, where
potential time savings are due to pruning a search subtree at the expense of
the time $t_2(n)$. However, whether outcome 2 takes place after a given state
is not known to the algorithm until the goal is found, and the algorithm must
decide whether to evaluate $h_2$ according to what it \textit{believes to be}
the probability of each of the outcomes. We derive a \textit{rational policy}
for when to evaluate $h_2$, under the myopic assumption that the algorithm
continues to behave like \lazyastar~afterwards (i.e., it will never again
consider bypassing the computation of $h_2$).

The time wasted by being sub-optimal in deciding whether to evaluate $h_2$ is
called the {\em regret} of the decision. If $h_2(n)$ is not helpful and we decide to compute it, the effort for evaluating $h_2(n)$ turns out to be wasted. On the other hand, if $h_2(n)$ is helpful but we decide to bypass it, we needlessly expand $n$. Due to the myopic assumption, \rationallazyastar~would evaluate both $h_1$ and $h_2$ for all successors of
$n$.

\begin{table}[t]
\begin{small}
\begin{center}
\begin{tabular}{|l|c|c|}
\hline
               & Compute $h_2$ & Bypass $h_2$\\
\hline
$h_2$ helpful &   0            & $t_e+(b(n)-1)t_d$\\
\hline
$h_2$ not helpful & $t_d$      & 0 \\
\hline
\end{tabular}
\end{center}
\end{small}\vspace{-0.2cm}
\caption{Time losses in Rational Lazy A*}
\label{tbl:rational-lazy-a-time}
\end{table}

Table~\ref{tbl:rational-lazy-a-time}
summarizes the regret of each possible decision, for each possible future
outcome; each column in the table represents a decision, while each row
represents a future outcome.
In the table, $t_d$ is the to time compute $h_2$ and re-insert $n$ into
\OPEN~thus delaying the expansion of $n$, $t_e$ is the time to remove $n$ from \OPEN,
expand $n$, evaluate $h_1$ on each of the $b(n)$ (``local branching factor'')
children $\{n'\}$ of $n$, and insert $\{n'\}$ into the open list.
Computing $h_2$ needless ``wastes'' $t_d$ time.
Bypassing $h_2$ computation when $h_2$ would have been helpful ``wastes''
$t_e+b(n)t_d$ time, but because computing $h_2$ would have cost $t_d$, the
regret is $t_e+(b(n)-1)t_d$.

Let us denote the probability that $h_2$ is helpful by
$p_h$.
%Due to line 2 in the table,
The expected regret of computing $h_2$ is thus $(1-p_h) t_d$.
%Due to line 3 in the table,
On other hand, the expected regret of bypassing $h_2$ is $p_h(
t_e+(b(n)-1)t_d)$. As we wish to minimize the expected regret, we should thus evaluate $h_2$ just when:
\begin{equation}
(1-p_h) t_d < p_h (t_e+(b(n)-1)t_d)
\end{equation}
or equivalently:
\begin{equation}
(1-b(n) p_h) t_d < p_h t_e
\end{equation}

If $p_h b(n) \ge 1$, then the expected regret is minimized by always
evaluating $h_2$, regardless of the values of $t_d$ and $t_e$.
In these cases, RLA* cannot be expected to do better than \lazyastar.
For example, in the 15-puzzle and its variants, the
effective branching factor is $\approx 2$. Therefore, if $h_2$ is expected to be helpful for more than half of the nodes $n$
on which \lazyastar~evaluates $h_2(n)$, then one should simple use \lazyastar.

For $p_h b(n) < 1$,  the decision of whether to evaluate $h_2$
depends on the values of $t_d$ and $t_e$:
\begin{equation}
\mbox{\bf evaluate }h_2\mbox{ \bf if }t_d<\frac {p_h} {1-p_hb(n)} t_e
\label{eqn:criterion-general}
\end{equation}
Let us analyze $t_d$ and $t_e$. Denote by
%$t_o$  the time
%to insert a single node into (or remove it from) \OPEN,
$t_c$ the time to generate the children of $n$. Then we have:
\begin{align}
t_d&=t_2+t_o\nonumber\\
t_e&=t_o + t_c+b (n) t_1 + b(n) t_o
\label{eqn:t-del-exp-expanded}
\end{align}
By substituting
(\ref{eqn:t-del-exp-expanded}) into (\ref{eqn:criterion-general}), obtain: {\bf evaluate} $h_2$ {\bf if}:
\begin{equation}
{t_2+t_o}<\frac {p_h \left[{t_c} + b (n)t_1+(b(n)+1){t_o}\right]} {1-p_hb(n)}
\label{eqn:criterion-expanded}
\end{equation}
The factor $\frac {p_h} {1-p_hb(n)}$ depends on the potentially unknown
probability $p_h$, making it difficult to reach the optimum decision.
However, if our goal is just to do better than \lazyastar, then it is safe to replace $p_h$ by an upper bound on $p_h$.

We now turn to implementation-specific estimation of the respective runtimes.
\OPEN~in \astar~is frequently implemented as a priority queue, and thus we have, approximately,
$t_o=\tau \log N_o$ for some $\tau$, when the size of \OPEN~is $N_o$.
Evaluating $h_1$ is cheap in many domains, as is the
case with MD in discrete domains, $t_o$ is the most significant part of
$t_{e}$. In such cases,
rule (\ref{eqn:criterion-expanded}) can be approximated as (\ref{eqn:criterion-gamma}):
\begin{align}
  &\mbox{\bf evaluate }h_2\mbox{ \bf if } t_2 < \frac {\tau p_h} {1-p_hb(n)} (b(n)+1)\log N_o
\label{eqn:criterion-gamma}
\end{align}
Rule (\ref{eqn:criterion-gamma})
recommends to evaluate $h_2$ mostly at late stages of the search,
when the open list is large, and in nodes with a higher branching factor.

In other domains, such as planning, both $t_1$ and $t_2$ are
significantly greater than both $t_o$ and $t_c$, and terms
not involving $t_1$ or $t_2$ can be dropped from
(\ref{eqn:criterion-expanded}), resulting in Rule (\ref{eqn:criterion-beta}):
\begin{align}
  &\mbox{\bf evaluate }h_2\mbox{ \bf if } \frac{t_2}{t_1} < \frac {p_hb(n)} {1-p_hb(n)}
\label{eqn:criterion-beta}
\end{align}
The right side of (\ref{eqn:criterion-beta}) grows with $b(n)$, and here it is beneficial to evaluate $h_2$
only for nodes with a sufficiently large branching factor. On rearranging equation \ref{eqn:criterion-beta},
we get the criterion which we actually use for planning domains,
which is to evaluate $h_2$ only when:
\begin{equation}
b(n) > \frac{t_2}{t_1 p_h (1 + \frac{t_2}{t_1})}
\label{eq:planning-rule}
\end{equation}

\subsection{Empirical results: weighted 15 puzzle}


\begin{table}
\begin{centering}
\begin{small}
\begin{tabular}{|c|| r r || r r r r || r r r r r | } \hline
&\multicolumn{2}{|c||}{\astar}&\multicolumn{4}{c||}{\lazyastar}&\multicolumn{5}{c|}{\rationallazyastar}\\
\hline
lookahead & generated & time & generated & Good1 & $h_2$ & time & generated & Good1   & Good2  & $h_2$     & time \\ \hline
         6 & 889,930  & {\bf 0.601}  & 889,930  & 257,598 & 632,332 & 0.462   & 944,750 &  299,479 & 239,320 &  405,951   & 0.446  \\ \hline
         8 & 740513  & 0.700  & 740,513  & 197,107 & 543,406 & {\bf 0.431}   & 892,216 &  233,370 & 303,655 &  260,823   & 0.402  \\ \hline
         10 & 612010 & 0.929  & 612,010  & 145,687 & 466,327 & 0.474   & 859,220 &  278,431 & 445,846 &  134,943   & {\bf 0.378}  \\ \hline
         12 & 454171 & 1.128  & 454,171  & 95,118  & 359,053 & 0.621   & 807,846 &  277,783 & 428,686 &  101,377   & 0.465  \\ \hline
\end{tabular}
\end{small}
\end{centering}\vspace{-0.3cm}
\caption{Weighted 15 puzzle: comparison of $A^*_{\max}$, Lazy $A^*$, and Rational Lazy $A^*$}\vspace{-0.3cm}
\label{tbl:rational-lazy-a-star}
\end{table}

In the uniform-cost 15 puzzle, the open list contains only a few different f-costs,
and is frequently implemented using buckets, violating the assumption of logarithmic time for which RLA* is beneficial. In order to
better evaluate the latter, we therefore use
the weighted 15-puzzle variant~\cite{thayer:bss}, where the
cost of moving each tile is equal to the number on the tile.  For consistency
of comparison, we used a subset of 36 problem instances out of the set of 100
15-puzzle instances by~\cite{BFID85}, keeping the problems which could
be solved with 2Gb of RAM and 15 minutes timeout using the Weighted Manhattan
heuristic (WMD) for $h_1$. As the second, expensive and informative, $h_2$ heuristic for \lazyastar~and \rationallazyastar, we a heuristic based
on lookaheads~\cite{DBLP:conf/aaai/SternKFH10}. Given a bound $d$ we applied a bounded depth-first search from a node $n$ and backtracked when we reached leaf nodes $l$ for which $g(l)+WMD(l)> g(n)+WMD(n)+d$. $f$-values from leaves were propagated to $n$.




Table~\ref{tbl:rational-lazy-a-star} presents the results averaged on all instances solved. The running times are reported relative to
the time of \astar with WMD (with no lookahead), which generated 2012643 nodes (not reported in the table). The first 3 columns of Table
\ref{tbl:rational-lazy-a-star} shows the results for \astar~with the lookahead
heuristic for different lookahead depths. The best time is achieved
for lookahead 6, (0.601 compared to \astar~with WMD). The fact that the time is not continuing to decrease with deeper lookaheads is clearly due to the fact that although the resulting heuristic improves as a function of lookahead depth (expanding and generating fewer nodes), the increasing overheads of computing the heuristic eventually outweights the computation time it saved by expanding fewer nodes.


The next 4 columns show the results for \lazyastar~
with WMD as $h_1$, lookahead as
$h_2$, for different lookahead depths.  The {\em Good1} column presents the number of nodes where \lazyastar saved the computation of $h_2$ while the $h_2$ column presents the number of nodes where $h_2$ was computed. $\approx 28\%$ of nodes were {\em Good1} and since $t_2$ was the most dominant time cost, most of this saving is reflected in the timing results.  The best results are achieved for lookahead 8, with a runtime of 0.431 compared to \astar~ with WMD.

The final columns show the results of RLA* with different lookaheads for
empirically best values of $\tau$. The {\em Good2} column counts the number of times that RLA* decided to bypass the $h_2$ computation and expand the node right away.Observe that RLA* outperforms \lazyastar, which in turn outperforms standard A*, for all lookahead depths. The lowest time with RLA* (0.378 of \astar with WMD) and empirically best $\tau$  was obtained for lookahead 10. That is achieved because the more expensive $h_2$ heuristic is computed less often, reducing its effective computational overhead, with little or no adverse effect in the number of expanded nodes. Although LA* expanded fewer nodes, RLA* performed much fewer $h_2$ computations as can be seen in the table, resulting in decreased overall runtimes for all lookahead depths.

