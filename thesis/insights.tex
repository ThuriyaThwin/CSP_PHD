In this section, we summarize our experience of applying rational
metareasoning to search problems. The recommendations provided here
should be treated as general guidelines to improving search algorithms
rather than as strict instructions. Designing better algorithms is 
an art, even if base on the solid foundation of rational
metareasoning.

\subsection{Assessing Applicability of Rational Metareasoning}

The first question raised by a metareasoning researcher facing a new
problem is whether a solution \emph{would benefit at all} from
applying rational metareasoning. In most cases, an existing algorithm
involving heuristic evaluation of search states justifies the attempt
to perform the heuristic computations selectively, based on the value
of information. However, there are two important exceptions:
\begin{itemize}
\item On the one hand, if \textbf{the heuristic relies on pre-computed
  data}, and online evaluation is cheap at the expense of intensive
  offline computation performed ahead of time and usable with a wide
  range of problem instances, computing the heuristic selectively is
  unlikely to save the total search time. For example, pattern
  databases \cite{pattern} proved to be an efficient approach for
  building informative and fast heuristics. 15-puzzle is one of
  domains in which pattern databases are particularly powerful
  \cite{Felner.apdb}; at the same time evaluating a state is just a
  small number of table lookups, so it never makes sense to try and
  compute the heuristic selectively.
\item On the other hand, in the case of an \textbf{informative but
  very expensive heuristic} evaluation of a state may
  actually make the search algorithm slower in most cases, and the
  algorithm should rely on domain-specific knowledge to
  identify the states in which computing the heuristic is benefitial. 
  Such fine domain-specific knowledge is usually hard to derive from
  a general notion of value of information. One example is
  high-accuracy solution counting algorithms, mentioned in
  Section~\ref{sec:cs-csp}, which are too time-consuming to be used
  in a value-ordering heuristic, unless a fine tuned higher-level
  heuristic is involved.
\end{itemize}
As a rule of thumb, rational metareasoning is beneficial for
optimizing heuristic evaluation in algorithms where:
\begin{enumerate}
\item Ubiquitous heuristic evaluation of the search space decreases the total search
  time.
\item The heuristic computation time constitutes a significant part of
  the total search time. 
\end{enumerate}
Of course, this criterion does not cover all possible situations where
rational metareasoning is helpful, but provides a good starting point
for assessing applicability of metareasoning approach to a given
problem domain.

\subsection{Identifying the Metareasoning Decision}


\subsection{Formulating an Utility and Information Model}

Selecting a simple model with the base non-metareasoning case.
Belief updating vs. frequentist approach.

\subsection{Analysing the Results of Empirical Evaluations}

Empirical evaluation is sensitive to correct emulation of real
algorithm conditions like relative times and heuristic performance
on test domains.

