In this section, we summarize our experience of applying rational
metareasoning to search problems. One should treat the recommendations
provided here as general guidelines to improving search algorithms
rather than as strict instructions. Designing better algorithms is an
art, even when based on the solid foundation of rational
metareasoning.

\subsection{Assessing Applicability of Rational Metareasoning}

The first question raised by a metareasoning researcher facing a new
problem is whether a solution \emph{would benefit at all} from
applying rational metareasoning. In most cases, an existing algorithm
involving heuristic evaluation of search states justifies an attempt
to perform the heuristic computations selectively, based on the value
of information. However, there are two important exceptions:
\begin{itemize}
\item On the one hand, if \textbf{the heuristic relies on pre-computed
  data}, and online evaluation is cheap at the expense of intensive
  offline computation performed ahead of time for a wide
  range of problem instances, computing the heuristic selectively is
  unlikely to save the total search time. For example, pattern
  databases \cite{pattern} proved to be an efficient approach for
  building informative and fast heuristics. 15-puzzle is one of
  domains in which pattern databases are particularly powerful
  \cite{Felner.apdb}; at the same time evaluating a state is just a
  small number of table lookups, so it never makes sense to try and
  compute the heuristic selectively.
\item On the other hand, in the case of an \textbf{informative but
  very expensive heuristic} evaluation of a state may
  actually make the search algorithm slower in most cases, and the
  algorithm should rely on domain-specific knowledge to
  identify the states in which computing the heuristic is beneficial. 
  Such fine domain-specific knowledge is usually hard to derive from
  a general notion of value of information. One example is
  high-accuracy solution counting algorithms, mentioned in
  Section~\ref{sec:cs-csp}, which are too time-consuming to be used
  in a value-ordering heuristic, unless a fine tuned higher-level
  heuristic is involved.
\end{itemize}
As a rule of thumb, rational metareasoning is beneficial for
optimizing heuristic evaluation in algorithms where:
\begin{enumerate}
\item Ubiquitous heuristic evaluation of the search space \emph{decreases} the total search
  time.
\item The heuristic computation time constitutes \emph{a significant part} of
  the total search time. 
\end{enumerate}
Of course, this criterion does not cover all possible situations where
rational metareasoning is helpful, but provides a good starting point
for assessing applicability of metareasoning approach to a given
problem domain.

\subsection{Identifying the Metareasoning Decision}

For a successful realization of the rational metareasoning layer it is
important to identify the metareasoning decision. Determining the
choices available at the decision points, and benefits and costs
associated with each of the choices is not always simple. For example,
in Section~\ref{sec:cs-mcts}, ``VOI-aware MCTS'', the originally
considered decision was whether to continue sampling or to commit to a
move. This model facilitated derivation of VOI estimates for simple
regret in Multi-armed bandits
(Section~\ref{sec:mcts-approx-nonbayesian-section}). However, we could
improve the Go agent only after realizing that for sampling in trees
the alternative is to either continue sampling or to commit to a move
\emph{and possibly spend the unused samples at the selected child} of
the current state (Section~\ref{sec:mcts-control-redistribution}).

Similarly, in \rationallazyastar~(Section~\ref{sec:cs-rla}) the
decision is whether to compute the expensive heuristic $h_2$ in the current
node $n$ or to bypass the evaluation \emph{and compute $h_2$ in all
  children of $n$}, as though the algorithm resorts to plain \lazyastar.
Earlier attempts to derive a metareasoning rule that compares the
benefit of pruning the subtree due to computing $h_2$ 
with the cost of traversing the subtree did not result in a faster
algorithm. 

One way to ensure robustness of an algorithm ${\mathcal A}^{\textrm MR}$ with the
metareasoning level is to start with an efficient non-metareasoning
algorithm ${\mathcal A}$ and implement metareasoning decisions in ${\mathcal
A}^{\textrm MR}$ as choices
between either behaving provably better than ${\mathcal A}$ or
resorting to ${\mathcal A}$.
\rationallazyastar~in Section~\ref{sec:cs-rla} which is based on
\lazyastar, and MAC with VOI-driven solution-counting value-ordering
heuristic~in Section~\ref{sec:cs-csp} which improves on MAC with
ubiquitous solution counting are examples of this technique.

\subsection{Formulating an Utility and Information Model}

Decision-making of the rational metareasoning approach is based on the
notions of state utility and value of information of computational and
base-level actions. A more complicated utility function does not necessarily
bring additional  benefits:
\begin{itemize}
\item A complicated function is more expensive to compute, and
  metareasoning computations tend to increase the total search
  time.  Chapter~\ref{ch:raticomp}, ``Rational Computation of Value of
  Information'' suggests some ways to leverage the costs of
  computations at the metareasoning level when an expensive utility
  function is absolutely necessary, but, as the case studies in
  Sections~\ref{sec:cs-csp}--~\ref{sec:cs-rla} show, even simple utility
  functions result in considerable improvements.
\item A utility model is often overly complicated because the
  algorithm designer tried to squeeze into the model domain-specific
  heuristic knowledge. However, metareasoning decisions
  that implicitly depend on heuristic knowledge impair the idea of
  rational metareasoning as an abstraction and hinder reasoning about
  the algorithm.
\end{itemize}

Estimating the value of information of actions involves maintaining
beliefs about probabilities of action outcomes. In some cases, such as
applying \rationallazyastar~to planning problems
\cite{TolpinEtAl.rla}, guessing prior beliefs and updating them based
on obtained evidence is possible and even necessary. In other cases,
such as the MAC algorithm for constraint satisfaction problems
(Section~\label{sec:cs-csp}), the beliefs can be constructed from the
analysis of the algorithm under assumptions about the structure of the
search space. Yet in other applications, such as the selection problem
(Section~\ref{sec:cs-mcts}), prior beliefs are not easily obtainable. An
alternative to maintaining beliefs is estimating the value of
information based on distribution-independent bounds derived from on
concentration inequalities;
Section~\ref{sec:mcts-approx-nonbayesian-section} serves as an example  
of this approach. An additional benefit of bound-based VOI estimates 
is robustness of metareasoning decisions which would be otherwise
by inadequate prior beliefs. 

\subsection{Analyzing the Results of Empirical Evaluation}

Empirical evaluation is an indispensable part of developing an
application of the rational metareasoning approach to a given
problem. Analysis of the empirical performance of the algorithm
facilitates verifying and improving the metareasoning model.  Still,
misleading results are frequent in the assessment of efficiency of the
rational metareasoning approach. Such assessment involves comparing
real search times which are influenced by inevitable noise of the
computer facilities, but even more by features of a particular problem
domain or test set. 

For example, a useful comparison of VOI-aware MCTS
to UCT in the context of Computer Go (Section~\ref{sec:mcts-emp-go})
required configuring restricted versions of both players which are
strictly based on Monte Carlo sampling with minimum use of
position-dependent rules. Otherwise, the evaluation would compare
compatibility of the rules with the sampling scheme in each of the
agents rather than the advantage of one sampling scheme over the
other. Developing a Go agent which is both based on VOI-aware sampling
and wins over other state-of-the-art Go agents is an ambitious
objective. However, this objective is only loosely related to
verification of the rational metareasoning approach to Monte Carlo
tree search. 

Another example is the analysis of
\rationallazyastar~(Section~\ref{sec:rla-empirical}. Applying
\rationallazyastar to 15-puzzle and to planning problems commands
different decision rules, because in 15-puzzle the time required to
compute $h_1$ is negligible compared to the open list manipulation,
while in planning both heuristics are expensive to compute. 15-puzzle
could also be used, in addition to evaluation on established planning
problem sets, to analyze efficiency of the approach  in the
planning domain using the appropriate decision rule, but in the latter
case the empirical results had to be processed accordingly,
by extracting just the heuristic computation times. 
