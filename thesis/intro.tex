Search for optimal or near optimal solutions of combinatorial problems
is a fundamental field of research in classical and applied Artificial
Intelligence. Search problems can be found in
navigation~\cite{Kocsis.uct}, routing~\cite{Bellman.routing},
planning~\cite{Domshlak.maxornot}, game
playing \cite{Braudis.pachi,Good.chess} and numerous other
combinatorial domains~\cite{Korf.partitioning}.  Despite the
tremendous volume of research on search in various settings, it is
well known that for many search problems no algorithm dominates all
others in all cases~\cite{Wolpert.nofree}. Similarly, many algorithms
have tunable parameters that need
optimizing~\cite{Hutter.spo,TolpinShimony.raticomp}. The nature of the
required solution (e.g., optimal, suboptimal etc.) also has
significant impact on the desired search algorithm. Selecting the best
algorithm and parameters for the task at hand is thus a non-trivial
and important issue~\cite{Allen.selheur,Gomes.portfolio}. Moreover,
every problem instance induces a search graph, which can be extremely
non-uniform and have areas with radically different behavior. This
calls for algorithms which change their settings dynamically during
the search, so as to better adapt to the task at hand.

In this research, a two-level framework for search is proposed: the
regular search level and a metareasoning level (MR). The MR level
selects the next computational operator to be used in the current
state of the search. The search level applies the selected
operator. This dual process is repeated until the search task is
completed, resulting in more flexible search algorithms that should
perform better than existing inflexible algorithms.  \textit{Rational
metareasoning}, a theory presented over two decades ago by Russell and
Wefald~\cite{Russell.right} aims at optimal decision-making
w.r.t. which computational operator should be applied at every point
of the search. It achieves a search algorithm that is (theoretically)
optimal in its use of resources, with self-adaptation capabilities to
the task and domain at hand. However, this rational MR theory is
extremely non-trivial to apply in actual search, due to the difficulty
in obtaining the requisite quantitative model (utility values and
probability distribution), and intractability of the
model~\cite{Conitzer.complexity}---often higher than that of the
search problem itself!  As a result, this theory has seen relatively
little application to real search problems to date.

This research develops a methodology of applying the rational
metareasoning theory to actual search problems.  On the one hand, some
of the theoretical difficulties are lifted; in particular, the problem
of efficiency of estimating the value of information of computational
actions is considered (Chapter~\ref{ch:raticomp}). On the other hand,
a series of case studies explores rational metareasoning extensions
for state-of-the-art search algorithms, and demonstrates how a
significant gain in the algorithm performance is achieved due to the
application of rational metareasoning. The cases studies investigate
various aspects of the methodology in the context of constraint satisfaction
problems (Chapter~\ref{ch:cs-csp}), as well as of approximation
(Chapter~\ref{ch:cs-mcts}) and optimal (Chapter~\ref{ch:cs-rla})
search in optimization problems. In addition to providing a
metareasoning-based solution to a particular problem, the studies
outline both ubiquitous aspects of application of rational metareasoning
(identifying base-level and computational actions, defining the
utility and the value of information, etc.)  and techniques which
exploit features present only in some problem classes, such as
estimating the value of information based on distribution-independent
bounds (Chapter~\ref{ch:cs-mcts}).

A straightforward application of rational metareasoning often faces
difficulties because some of the assumptions of the original approach do not
hold. For example, estimating the value of information depends on
the belief distribution of action outcomes; however, a reasonable choice
of the distribution is not always possible. Chapter~\ref{ch:cs-csp}
derives the distribution model from the properties of the search
algorithm rather than from the set of problem instances. Another
example is the myopic assumption \cite{Russell.right} which suggests
that a computational action can be chosen based on its anticipated immediate
effect. In some search algorithms, the information is obtained through
\emph{sampling}---recurring actions with probabilistic outcomes---and
the value of information of a single sample is often
zero. Chapter~\ref{ch:cs-mcts} addresses such a case in the context
of Monte-Carlo tree search and suggests a value of information
estimate based on recurring actions of the same kind.

Advanced search algorithms are often parameterized and require tuning
to achieve the best performance. Building an algorithm upon the
principles of rational metareasoning allows to keep the number of
tunable parameters to a minimum---a single parameter needs to be tuned
in Chapters~\ref{ch:cs-csp}~and~\ref{ch:cs-rla}. In addition, as
Chapter~\ref{ch:cs-csp} demonstrates, parameters
of a metareasoning-based algorithm are likely to reflect
implementation details of the algorithm and the heuristics rather
than features of a particular set of problem instances; this is
advantageous since the algorithm can be tuned on a small training set,
and readjustments for different sets of problem instances are
unnecessary. 

Finally, rational metareasoning is commonly applied to approximating
search where decision quality is traded off for computation time. Optimal
search algorithms aim at finding the optimal solution in the shortest
possible time, and seemingly little can be achieved through
application of rational metareasoning.  Chapter~\ref{ch:cs-rla}
presents metareasoning-based improvements to a variant of
the~\astar~algorithm which result in finding the optimal solution in a
shorter time using the same heuristic functions; the employed
technique is general enough to extend to other optimal
search algorithms.

The rest of the thesis is organized as follows. Chapter~\ref{ch:bg}
provides the necessary background information about the rational
metareasoning approach as well as about search problems and
algorithms. Chapter~\ref{ch:ramesrch} describes application of the
rational metareasoning approach to search problems and discusses
difficulties arising in design of search algorithms based on the
approach. Chapter~\ref{ch:raticomp}, based
on \cite{TolpinShimony.raticomp}, introduces rational computation of
value of information---an important issue in design of efficient
search algorithms based on rational metareasoning. Case studies of the
rational metareasoning approach in several search problems are
presented and evaluated in Chapters~\ref{ch:cs-csp}--\ref{ch:cs-rla} (based
on~\cite{TolpinShimony.csp,TolpinShimony.mcts,HayRussellTolpinShimony.selecting,TolpinEtAl.rla}).
Chapter~\ref{ch:insights} summarizes the experience of applying 
the rational metareasoning approach to various search problems, and
Chapter~\ref{ch:summary} concludes this thesis with a discussion of
achieved results and further research directions.

