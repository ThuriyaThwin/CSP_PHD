A common approach in Artificial intelligence (AI)
solves computational problems by designing {\em agents}. Agents
act in an environment, exploring and possibly modifying the
environment in order to reach the {\em goal}: a solution of the
problem. Agent behavior is determined by the {\em agent function} that
maps the agent's knowledge about the environment to actions. AI
classifies problems according to the number of agents, the type of
interaction among the agents and between the agents and the
environment, as well as by the {\em problem domain}---properties of
the environment which help design efficient problem-specific agents.

A simple yet large class of problems is {\em problem-solving
  search}. In problem-solving search, a single agent acts in a neutral
environment to reach the goal. Many problems, such as routing and
path-finding problems, finite-domain constraint satisfaction, function
optimization, fit within the problem-solving search abstraction. The
problem-solving search agent's behavior is described by a {\em search
  algorithm}. The same search algorithms can be used to solve
different search problems, but a general search algorithm often
requires adaptation to solve a particular problem efficiently.

Computer scientists approach problem-solving search in two ways. On
the one hand, general search algorithms capable of solving large sets
of search problems are designed, such as A* search for path-finding
problems, or the Maintaing Arc Consistency (MAC) algorithm for
finite-domain constraint satisfaction \cite{Russell.aima}. Theoretical
performance bounds can often be proved for these algorithms, and the
same algorithm can efficiently solve many problem instances with
little or no adaptation. On the other hand, the algorithm performance
can be significantly improved by tuning the algorithm to a particular
problem domain. For example, specialized algorithms for task
scheduling outperform general constraint satisfaction algorithms by
orders of magnitude \cite{Wolf.scheduling}. However, such fine-tuned
algorithms exhibit good performance only on small sets of search
problems, and the effort invested in the algorithm design cannot be
reused in other problem domains.

Specialized versions of general search algorithms are often created by
combination and selective application of {\em search
  heuristics}---domain-specific modifications or extensions of search
algorithms. A human expert decides which heuristics to use with the
problem domain, and specifies how the search algorithm should apply
the heuristics to solve a particular problem instance. A search
algorithm that rationally select and applies heuristics would decrease
the need for the costly human expertise.

The principles of rational metareasoning \cite{Russell.right} can be
used to design rational search agents. A rational agent deliberates
before acting. The deliberation is composed of {\em computational
  actions}. The purpose of the deliberation is to select the best {\em
  base-level action}. A base-level action affects the state of the
search algorithm looking for a solution of a problem instance, for
example, by a exploring the search space, or pruning a part of the
search that does not contain a solution. Computational actions are
selected according to their net value of information, the difference
between the expected benefit and the cost of the action, and the agent
deliberates only if there is a computational action with positive
value of information.

Some rational search algorithms were designed and shown to compare to
or even outperform manually tuned algorithms
\cite{Russell.right}. However, wide adoption of rational metareasoning
algorithms for problem solving search is hindered both by theoretical
difficulties and by lack of problem domain specific case studies. This
research aims at lifting some of the theoretical difficulties in
application of the rational methodology. In particular, the problem of
efficiency of estimating the value of information of computational
actions is considered. In addition, this research proposes rational
metareasoning extensions for several search algorithms, and
theoretically and empirically analyses the gain in performance due to
the extensions.

The presented case studies explore rational
metareasoning in the context of satisfaction problems
(Chapter~\ref{ch:cs-csp}), as well as of approximating
(Chapter~\ref{ch:cs-mcts}) and optimal (Chapter~\ref{ch:cs-rla})
search in optimization problems. In addition to providing a
metareasoning-based solution to a particular problem, the studies
outline both techniques which are common to different problem types
(identifying base-level and computational actions, 
defining the utility and the value of information, etc.)
and those which exploit features present only in some of the problems,
such as estimating the VOI based on distribution-independent bounds
(Chapter~\ref{ch:cs-mcts}). 

A straightforward application of rational metareasoning often faces
difficulties because some of the assumptions of the approach do not
hold. For example, estimating the value of information depends on
the belief distribution of action outcomes; however, a reasonable choice
of the distribution is not always possible. Chapter~\ref{ch:cs-csp}
derives the distribution model from the properties of the search
algorithm rather than from the set of problem instances. Another
example is the myopic assumption \cite{Russell.right} which suggests
that an action can be chosen based on its anticipated immediate
effect. In some search algorithms, the information is obtained through
\emph{sampling}---recurring actions with probabilistic outcomes---and
the myopic VOI of a single sample is often
zero. Chapter~\ref{ch:cs-mcts} addresses such a case in the context
of Monte-Carlo tree search and suggests a semi-myopic VOI estimate.

Advanced search algorithms are often parameterized and require tuning
to achieve the best performance. Building an algorithm upon the
principles of rational metareasoning allows to keep the number of
tunable parameters to a minimum---a single parameter needs to be tuned
in Chapters~\ref{ch:cs-csp},~\ref{ch:cs-rla}. In addition, as
Chapter~\ref{ch:cs-csp} demonstrates, parameters
of a metareasoning-based algorithm are likely to reflect
implementation details of the algorithm and the heuristics rather
than features of a particular set of problem instances.

Finally, rational metareasoning is commonly applied to approximating
search where decision quality is traded for computation time. Optimal
search algorithms aim at finding the optimal solution in the shortest
possible time, and seemingly little can be achieved through
application of rational metareasoning.  Chapter~\ref{ch:cs-rla}
presents metareasoning-based improvements to a variant of
the~\astar~algorithm which result in finding the optimal solution in a
shorter time using the same heuristic functions; the employed
technique is general enough to extend to other optimal
search algorithms.

The rest of the thesis is organized as follows. Chapter~\ref{ch:bg}
provides the necessary background information about the rational
metareasoning approach as well as about search problems and
algorithms. Chapter~\ref{ch:ramesrch} describes application of the
rational metareasoning approach to search problems and discusses
difficulties arising in design of search algorithms based on the
approach. Chapter~\ref{ch:raticomp}, based
on \cite{TolpinShimony.raticomp}, introduces rational computation of
value of information---an important issue in design of efficient
search algorithms based on rational metareasoning. Case studies of the
rational metareasoning approach in several search problems are
presented and evaluated in Chapters~\ref{ch:cs-csp}--\ref{ch:cs-rla} (based
on~\cite{TolpinShimony.csp,TolpinShimony.mcts,HayRussellTolpinShimony.selecting,TolpinEtAl.rla}).
Chapter~\ref{ch:insights} summarizes the experience of applying 
the rational metareasoning approach to different search problems, and
Chapter~\ref{ch:summary} concludes this thesis with a discussion of
achieved results and further research directions.

