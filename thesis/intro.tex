A common approach in Artificial intelligence (AI)
solves computational problems by designing {\em agents}. Agents
act in an environment, exploring and possibly modifying the
environment in order to reach the {\em goal}: a solution of the
problem. Agent behavior is determined by the {\em agent function} that
maps the agent's knowledge about the environment to actions. AI
classifies problems according to the number of agents, the type of
interaction among the agents and between the agents and the
environment, as well as by the {\em problem domain}---properties of
the environment which help design efficient problem-specific agents.

A simple yet large class of problems is {\em problem-solving
  search}. In problem-solving search, a single agent acts in a neutral
environment to reach the goal. Many problems, such as routing and
path-finding problems, finite-domain constraint satisfaction, function
optimization, fit within the problem-solving search abstraction. The
problem-solving search agent's behavior is described by a {\em search
  algorithm}. The same search algorithms can be used to solve
different search problems, but a general search algorithm often
requires adaptation to solve a particular problem efficiently.

Computer scientists approach problem-solving search in two ways. On
the one hand, general search algorithms capable of solving large sets
of search problems are designed, such as A* search for path-finding
problems, or the Maintaing Arc Consistency (MAC) algorithm for
finite-domain constraint satisfaction \cite{Russell.aima}. Theoretical
performance bounds can often be proved for these algorithms, and the
same algorithm can efficiently solve many problem instances with
little or no adaptation. On the other hand, the algorithm performance
can be significantly improved by tuning the algorithm to a particular
problem domain. For example, specialized algorithms for task
scheduling outperform general constraint satisfaction algorithms by
orders of magnitude \cite{Wolf.scheduling}. However, such fine-tuned
algorithms exhibit good performance only on small sets of search
problems, and the effort invested in the algorithm design cannot be
reused in other problem domains.

Specialized versions of general search algorithms are often created by
combination and selective application of {\em search
  heuristics}---domain-specific modifications or extensions of search
algorithms. A human expert decides which heuristics to use with the
problem domain, and specifies how the search algorithm should apply
the heuristics to solve a particular problem instance. A search
algorithm that rationally select and applies heuristics would decrease
the need for the costly human expertise.

The principles of rational metareasoning \cite{Russell.right} can be
used to design rational search agents. A rational agent deliberates
before acting. The deliberation is composed of {\em computational
  actions}. The purpose of the deliberation is to select the best {\em
  base-level action}. A base-level action affects the state of the
search algorithm looking for a solution of a problem instance, for
example, by a exploring the search space, or pruning a part of the
search that does not contain a solution. Computational actions are
selected according to their net value of information, the difference
between the expected benefit and the cost of the action, and the agent
deliberates only if there is a computational action with positive
value of information.

Some rational search algorithms were designed and shown to compare to
or even outperform manually tuned algorithms
\cite{Russell.right}. However, wide adoption of rational metareasoning
algorithms for problem solving search is hindered both by theoretical
difficulties and by lack of problem domain specific case studies. This
research aims at lifting some of the theoretical difficulties in
application of the rational methodology. In particular, the problem of
efficiency of estimating the value of information of computational
actions is considered. In addition, this research proposes rational
metareasoning extensions for several search algorithms, and
theoretically and empirically analyses the gain in performance due to
the extensions.

The rest of the research proposal is organized as
follows. Chapter~\ref{ch:bg} provides the necessary background
information about the rational metareasoning approach as well as about
search problems and algorithms. Chapter~\ref{ch:ramesrch}
describes application of the rational metareasoning approach to 
search problems and discusses difficulties arising in design of
search algorithms based on the approach. Chapter~\ref{ch:raticomp} introduces 
rational computation of value of information---an important issue in
design of efficient search algorithms based on rational
metareasoning. Applications of the rational metareasoning approach to
several search problems are presented and evaluated in
Chapter~\ref{ch:applications}. Chapter~\ref{ch:summary} concludes this
research proposal with a summary of achieved results and further
research directions.