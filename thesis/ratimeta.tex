While ideally programs, or agents, should {\em act rationally}
\cite{Russell.aima}, absolute rationality is not feasible. The
computational power of the agent approaching the problem must be taken
into account. Rational metareasoning \cite{Russell.right} is an
approach to building {\em bounded optimal} agents, agents which find a
solution that is optimal given its computational resources
\cite{Horvitz.reasoningabout}. The approach describes a method of
choosing {\em meta-level actions} and is based on notions of {\em
value of information} and {\em time cost}.

\subsection{Meta-level Actions}

The agent performs a sequence of base-level actions drawn from a known
set $\{A_i\}$. Before committing to each action, the agent
deliberates, and the deliberation can be represented as a sequence of
meta-level actions from a set $\{S_j\}$. At any given time there is a
base-level action, $A_\alpha$, that appears best to the agent. The
goal of subsequent meta-level actions is to refine the choice of
$A_\alpha$ before performing the base-level action.

The agent selects $\alpha$ by numericaly estimating the {\em
  utilities} of action outcomes; according to decision theory, an
optimal action $A_\alpha$ is one which maximizes the agent's expected
utility:
\begin{eqnarray}
\label{eq:bg-limited-eu}
\IE[U(A_i)]&=&\sum_k P(W_k) U(A_i,W_k)\\
\alpha&=&\arg \max_i \IE[U(A_i)]
\end{eqnarray}
where $\{W_k\}$ is the set of possible world states, and $P(W_k)$ is
the belief probability that the world is currently in state $W_k$.

\subsection{Value of Information}
\label{sec:ratimeta-voi}

A meta-level action affects the choice of the base-level action
$A_\alpha$ by changing the belief distribution over the world
states. The {\em value} of a meta-level action is measured by the
resulting increase in the utility of $A_\alpha$. Since neither the
outcomes of meta-level actions nor the true utility of $A_\alpha$ are
known in advance, a meta-level action is selected according to its
expected influence on the expected utility of $A_\alpha$.

$S_j$ affects the internal state of the agent and the effect of
a possible further meta-level action sequence $\mathbf{T}$. Thus, the expected utility of
$S_j$ is:
\begin{equation}
\label{eq:bg-limited-su}
\IE[U(S_j)]=\sum_{\mathbf{T}}P(\mathbf{T})\IE[U(A_{\alpha}^\mathbf{T},S_j\cdot\mathbf{T})]
\end{equation}
where $S_j\cdot\mathbf{T}$ denotes the meta-level action $S_j$
followed by possible further meta-level action sequence $\mathbf{T}$.
The {\em value of information} of a meta-level action $S_j$ is
the expected difference between the expected utility of $S_j$ and the
expected utility of the current $A_\alpha$.
\begin{equation}
\label{eq:bg-limited-nv}
V(S_j)=\IE(\IE(U(S_j))-\IE(U(A_\alpha)))
\end{equation}

While a perfectly rational agent would always choose the most valuable
computation sequence, an agent with only limited rationality makes
decisions based on an approximation of the utility.

\subsection{Benefit and Time Cost}
\label{sec:ratimeta-benefit-timecost}

The general dependence on the overall state complicates the analysis. Under
certain assumptions, it is possible to capture the dependence of utility on time in a
separate notion of {\em time cost} $C$. Then, the utility of an action $A_i$ taken
 after a meta-level action $S_j$ is the utility of $A_i$ taken now less the
cost of time for performing $S_j$:

\begin{equation}
\label{eq:bg-limited-iu}
U(A_i, S_j) = U(A_i) - C(A_i, S_j)
\end{equation}

It is customary to call the current utility of a future base-level
action its {\em intrinsic utility}. The separation into intrinsic
utility and time cost allows to estimate the utility of a base-level
action in a time-independent manner, and then refine the net utility
estimate according to the time pressure represented by $C$.

In many cases, the time cost of an internal action is independent of
the subsequently taken base-level action. When $C$ depends only on
$S_j$, (\ref{eq:bg-limited-nv}) can be rewritten with the cost and the
intrinsic value of information of a computation as separate terms.

\begin{eqnarray}
\label{eq:bg-limited-v=bc}
V(S_j)&=&\IE\left(\IE(U(A_\alpha^j, S_J))-\IE(U(A_\alpha))\right)\nonumber\\
     &=&\IE\left(\IE(U(A_\alpha^j))-\IE(U(A_\alpha))\right)-C(S_j)\\
     &=& \Lambda(S_j)-C(S_j)
\end{eqnarray}
where
\begin{equation}
\label{eq:bg-limited-benefit}
\Lambda(S_j)=\IE\left(\IE(U(A_\alpha^j))-\IE(U(A_\alpha))\right)
\end{equation}
denotes the intrinsic value of information, that is, the expected
difference between the intrinsic expected utilities of the new
and the old selected base-level action, computed after the meta-level
action was taken.

For any particular outcome of the meta-level action, one of the
following cases takes place:

\begin{enumerate}
\item the selected base-level action $A_\alpha$ stays the same;
  consequently, $\IE(U(A_\alpha^j))-\IE(U(A_\alpha))$ is zero;
\item a different base-level action $A_\alpha^j$ is selected, with
  {\bf higher} expected utility than the expected utility of
  $A_\alpha$ before the meta-level action was taken;
\item a different $A_\alpha^j$ is selected, and its expected
  utility is {\bf the same} as or {\bf lower} than the expected
  utility of $A_\alpha$ before the meta-level action was taken.
\end{enumerate}

In the last two cases, the difference is positive---although the
expected utility of the final choice can decrease, the latter
choice appears to be better than the earlier one due
to the updated knowledge about all actions. Thus, while the
net value of information can be either positive or negative depending on
the cost of the action, the intrinsic value of information is always
non-negative.

