Principles of rational metareasoning were formulated by Russell
and Wefald \cite{Russell.right}, following Horvitz
\cite{Horvitz.reasoningabout}. In a case study of application of rational
metareasoning to problem-solving search, Russell and Wefald
\cite{Russell.right} described the design of the DTA* search algorithm,
a rational version of RTA* search by Korf \cite{Korf.rta}. 

Horvitz and Klein \cite{Horvitz.proving} applied rational
metareasoning to theorem proving. In particular, the authors showed how
decision-theoretic methods can be used to determine the value of
continuing to deliberate versus taking immediate action in
time-critical situations. In a later work, Horvitz et al
\cite{Horvitz.bayesian} described methods of the
decision-theoretic control of hard search and
reasoning algorithms, illustrating the approach with a focus on the
task of predicting run time for general and domain-specific solvers on
a hard class of structured constraint satisfaction problems. 

Zilberstein \cite{Zilberstein.PHD} employed limited rationality
techniques to analyze any-time algorithms. Radovilsky and Shimony
\cite{Radovilsky.oss} applied principles of rational metareasoning to
the design of any-time algorithms for observation subset
selection. Principles of rational metareasoning are used in some
multi-armed bandit algorithms \cite{Vermorel.bandits}.

Gomes and Selman \cite{Gomes.portfolio} analyzed dynamic algorithm
portfolios to hard combinatorial search problems and proposed
techniques for online algorithm selection and resource reallocation
based on algorithm performance profiles. Domshlak, Karpas and
Markovitch \cite{Domshlak.maxornot} presented a method of reducing the
cost of combining heuristics for optimal planning search by choosing
the best heuristic to compute at each search state.

Domain-specific search algorithms and heuristics are discussed in 
Chapter~\ref{ch:case-studies}: Case Studies. Work relevant to each of
the problem domains is cited in the corresponding sections. 
