The proposed research comprises two major directions:
\begin {enumerate}
\item Rational computation of value of information (Chapter~\ref{ch:raticomp}).
\item Case studies of application of rational metareasoning to
  selection and application of heuristic computations (Chapter~\ref{ch:applications}).
\end {enumerate}

The first direction considers cases in which estimating the value
of information of a computational action is expensive, as well as
cases of too many computational actions, such that although
estimating the value of information of a single action is cheap,
estimating the value of information of all actions at each step of the
algorithm is expensive. Some results of research of the case when
computational actions can be performed in any order
(Section~\ref{sec:raticomp-persistent-actions}) were published
\cite{Tolpin.raticomp}, but additional work is required both in
the theoretical analysis and in the empirical evaluation. Further
research of rational recomputation of value of information should
result in:
\begin{itemize}
\item Thorough theoretical analysis of the proposed algorithm for
rational computation of the value of information of ``persistent
actions'' (Section~\ref{sec:raticomp-persistent-actions}).
\item A version of the algorithm which re-uses VOI computations
across state transitions.
\item Theoretical and empirical analysis of the probabilistic model of
uncertainty about the value of information of computational actions.
\item An algorithm or algorithms for the case of very large or infinite
sets of simultanenously considered computational actions
(Section~\ref{sec:raticomp-infinite-spaces}).
\end{itemize}

The second direction considers several search problems and aims at
improving some well-known algorithms for solving the problems
using the rational metareasoning approach. Initial results in this
direction confirm importance of the rational metareasoning approach to
problem-solving search in applications.
 \begin{itemize}
\item Research results were published for parameter tuning with
inexact measurements in application to SVM parameter optimization and
metrology equipment setup \cite{TolpinShimony.blinkered}. The rational
algorithm for parameter tuning was further extended to incorporate
rational computation of value of information, and a comparative case
study of the basic and the extended algorithm was provided in
\cite{Tolpin.raticomp}.
\item A rational metareasoning model for the solution-counting
heuristic in constraint satisfaction was formulated, and initial
theoretical analysis and empirical evaluation of the model were
performed. Work is underway on preparing the results for publication.
\end{itemize}
Further work on applications of rational metareasoning to search
problems is planned as follows:
\begin{description}
\item[Parameter tuning (Section~\ref{sec:app-apt}):] algorithms for
  parameter optimization in large and high-dimensional search spaces
  should be designed. The algorithms for large spaces
  would rely on results of the research proposed in
  Section~\ref{sec:raticomp-infinite-spaces}.
\item[Counting based heuristics for constraint satisfaction problems
  (Section~\ref{sec:app-csp}):] the described metareasoning model
  (Section~\ref{sec:app-csp-rational}) should be theoretically analyzed
  and empirically evaluated, and feasible solution counting algorithms
  should be compared. Selection of heuristics depending on the
  likelihood of problem consistency should be explored.
\item[Remote sensing in the canadian traveller problem
  (Section~\ref{sec:app-ctp}):] CTP-SF should be formally
  stated as a selection problem. Theoretical performance bounds
  for a solution of the problem, or some partial cases of the problem, should be
  obtained. The solution should be empirically evaluated. An heuristic
  for the general problem of CTP with remote sensing should be designed
  based on the solution for CTP-SF.
\item[Learning Bayesian networks from large databases
  (Section~\ref{sec:app-bnlearn}):] Rational metareasoning algorithm for
  score-based BN structure learning algorithms should be designed. A model for
  the utility of such heuristics should be proposed, an a rational
  structure-learning algorithm should be evaluated and compared with
  known algorithms.
\end{description}

As a whole, the ongoing research should advance the use of rational
metareasoning in problem-solving search algorithms. Applications of
rational metareasoning in the case studies should serve as examples
and help researchers employ the methodology in solutions for other
problems. Advances in rational computation of VOI should increase
performance and applicability of existing and new search algorithms
and alleviate dependence of algorithm performance on manual
fine-tuning.