The thesis comprises two major topics:
\begin {enumerate}
\item Rational computation of value of information (Chapter~\ref{ch:raticomp}).
\item Case studies of application of rational metareasoning to
  selection and application of heuristic computations (Chapters~\ref{ch:cs-csp}--\ref{ch:cs-rla}).
\end {enumerate}

The first topic addressed cases in which estimating the value
of information of a computational action is expensive, as well as
cases of too many computational actions, such that although
estimating the value of information of a single action is cheap,
estimating the value of information of all actions at each step of the
algorithm is expensive. The research, published in
\cite{TolpinShimony.raticomp}, resulted in an improvement to a widely
used class of VOI-based optimization algorithms that allows to decrease the
computation time while only slightly affecting the reward. Theoretical
analysis of the proposed approach to rational computation of the
value of information was supported by empirical evaluation of several
combinations of algorithms and search problems.

The second topic considered several search problems and improved some
well-known algorithms for solving the problems using the rational
metareasoning approach:
\begin {itemize}
\item Adaptive deployment of value-ordering heuristics in constraint
  satisfaction problems (Chapter~\ref{ch:cs-csp}) \cite{TolpinShimony.csp};
\item Monte Carlo tree search based on simple regret
  (Chapter~\ref{ch:cs-mcts}) \cite{TolpinShimony.mcts,HayRussellTolpinShimony.selecting};
\item Decreasing heuristic evaluation time in a variant of A*
  (Chapter~\ref{ch:cs-rla}) \cite{TolpinEtAl.rla}. 
\end {itemize}
In addition to the algorithm improvements, the studies demonstrated a
number of common rational metareasoning techniques which can be 
extended to other problem types. In particular,
\begin{itemize}
\item Chapter~\ref{ch:cs-mcts}, ``VOI-aware Monte-Carlo tree search''
provided distribution-independent upper bounds for semi-myopic VOI
estimates in Monte-Carlo sampling.
\item Chapter~\ref{ch:cs-rla}, ``Towards rational deployment of Multiple
Heuristics in A*'', introduced a novel area of application of rational
metareasoning---optimal search in optimization problems.
\end{itemize}

As a whole, the research advanced the use of rational
metareasoning in problem-solving search algorithms. Applications of
rational metareasoning in the case studies serve as examples
to help researchers employ the methodology in solutions for other
problems. Advances in rational computation and estimation of VOI increase
performance and applicability of existing and new search algorithms
and alleviate dependence of algorithm performance on manual
fine-tuning.

The field of rational metareasoning still poses serious
challenges. An important yet unanswered question is the extent
to which algorithm performance can be improved due to employment of
the metareasoning approach. In the case studies presented in the thesis,
the improvements, while approached the theoretical limits of each
particular algorithm variant, were moderate. It is still not clearly
understood \textbf{whether a dramatic breakthrough in performance is possible}
if more elaborated models of utility and information are used, or the 
limitations are immanent to the approach itself. Another interesting
field of research is rational metareasoning where \textbf{action costs and
state utilities are not commensurable}. Instead of mapping between
different measures, as suggested in Chapter~\ref{ch:raticomp}, a
more general combination function should probably be used for best
results. Ways of choosing or deriving such a function still remain
to be discovered.
